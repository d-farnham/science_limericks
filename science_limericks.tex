\documentclass{article}
\usepackage{hyperref}


\title{
Limericks based on my scientific articles
}
\author{David J. Farnham}

\begin{document}
\maketitle




\noindent
\textbf{Zonal wind indices to reconstruct CONUS winter precipitation} (with Scott S and Manu L; \url{http://doi.wiley.com/10.1002/2017GL075959})\\

\vspace{0.25cm}

{\large
\noindent
\textit{When the tropical waters are warm,\\
Some warn that intense rainfall will swarm,\\
\null \hspace{0.5cm} But now we can glean,\\
\null \hspace{0.5cm} From the data we've seen,\\
There's no guarantee of a storm.}
}

\vspace{0.75cm}


\noindent
\textbf{Adaptation over Fatalism: Leveraging High-Impact Climate Disasters to Boost Societal Resilience} (led by James D-G with Manu L, and Michelle H; \url{https://ascelibrary.org/doi/10.1061/%28ASCE%29WR.1943-5452.0001190})\\

\vspace{0.25cm}

{\large
\noindent
\textit{Climate change may well be destructive,\\
And the urge to despair seductive,\\
\null \hspace{0.5cm} But to adapt be sure,\\
\null \hspace{0.5cm} That learning from days before,\\
Can still be quite constructive.}
}

\vspace{0.75cm}

\noindent
\textbf{Decadal scale climate variation presents risks and opportunities for managing wind/solar power supply} (in prep)\\

\vspace{0.25cm}

{\large
\noindent
\textit{When potential returns we are summing,\\
We guess the past foretells what is coming,\\
\null \hspace{0.5cm} But we may be optimistic,\\
\null \hspace{0.5cm} Based on the phase of the Pacific,\\
That the transmission lines may be humming.}
}

\newpage

\noindent
\textbf{Regional Extreme Precipitation Events: Robust Inference From Credibly Simulated GCM Variables} (with James D-G and Manu L; \url{https://doi.org/10.1002/2017WR021318})\\

\vspace{0.25cm}

{\large
\noindent
\textit{When precip we're inclined to project,\\
We obsess on which mods to select,\\
\null \hspace{0.5cm} But we should not forget,\\
\null \hspace{0.5cm} That other fields may yet,\\
Be more useful than precip direct.}
}

\vspace{0.75cm}


\noindent
\textbf{Predictive statistical models linking antecedent meteorological conditions and waterway bacterial contamination in urban waterways} (with Manu L; \url{http://linkinghub.elsevier.com/retrieve/pii/S004313541500113X})\\

\vspace{0.25cm}

{\large
\noindent
\textit{When the banks begin to urbanize,\\
Sewage and rainfall may fraternize,\\
\null \hspace{0.5cm} But if good vars are picked,\\
\null \hspace{0.5cm} A model may predict,\\
When contacting the water is wise.}
}

\vspace{0.75cm}

\noindent
\textbf{Citizen science-based water quality monitoring: Constructing a large database to characterize the impacts of combined sewer overflow in New York City} (with Rebecca G., Diana H., Wade M., Patricia C., Nina Z., Rob B.; \url{https://www.sciencedirect.com/science/article/pii/S0048969716325694})\\

\vspace{0.25cm}

{\large
\noindent
\textit{Science, a tedious endeavor,\\
But let’s not look past a prime lever,\\
\null \hspace{0.5cm} When it comes to fieldwork,\\
\null \hspace{0.5cm} Many hands make light work,\\
To collect data that’s forever.}
}


\vspace{0.75cm}

\noindent
\textbf{Developing Reliable Hourly Electricity Demand Data through Screening and Imputation} (led by Tyler R. and me, with Dan T. and Ken C.; in revision at \textit{Scientific Data})\\

\vspace{0.25cm}

{\large
\noindent
\textit{Much missing and suspect data; shoot!\\
But let us stay calm and resolute,\\
\null \hspace{0.5cm} For optimization,\\
\null \hspace{0.5cm} Without great frustration,\\
Merely compels us all to impute.}
}


\end{document}